%!TEX TS-program = xelatex
%!TEX encoding = UTF-8 Unicode

%-------------------------------------------------------------------------------
% CONFIGURATIONS
%-------------------------------------------------------------------------------
% A4 paper size by default, use 'letterpaper' for US letter
\documentclass[11pt, a4paper]{awesome-cv}

% Configure page margins with geometry
\geometry{left=1.4cm, top=.8cm, right=1.4cm, bottom=1.8cm, footskip=.5cm}

% Specify the location of the included fonts
\fontdir[fonts/]

% Color for highlights
\colorlet{awesome}{awesome-blue}

% Set false if you don't want to highlight section with awesome color
\setbool{acvSectionColorHighlight}{true}

% If you would like to change the social information separator from a pipe (|) to something else
\renewcommand{\acvHeaderSocialSep}{\quad\textbar\quad}

%-------------------------------------------------------------------------------
%	PERSONAL INFORMATION
%-------------------------------------------------------------------------------
% Available options: circle|rectangle,edge/noedge,left/right
\photo[rectangle,noedge,right]{profile}
\name{François}{Théroux}
\position{Consultant - Artificial Intelligence, Big Data and Cloud Computing}

\mobile{+1 (450) 400-1040}
\email{ftheroux@progranova.com}
\homepage{www.progranova.com}
\linkedin{ftheroux}

\quote{``My passion: Store, visualize and use data to create value in your corporation"}

%-------------------------------------------------------------------------------
\begin{document}

% Give optional argument to change alignment(C: center, L: left, R: right)
\makecvheader[L]

% Print the footer with 3 arguments(<left>, <center>, <right>)
\makecvfooter
  {Progranova - Copyright \textcopyright 2019}
  {François Théroux~~~·~~~Curriculum Vitae}
  {\thepage}

%-------------------------------------------------------------------------------
%	CV/RESUME CONTENT
%-------------------------------------------------------------------------------
%-------------------------------------------------------------------------------
%	SECTION TITLE
%-------------------------------------------------------------------------------
\cvsection{Skills}


%-------------------------------------------------------------------------------
%	CONTENT
%-------------------------------------------------------------------------------
\begin{cvskills}

%---------------------------------------------------------
  \cvskill
    {Programming}
	{Python, R, Java, Scala, C\#, C/C++, JavaScript, SQL, NoSQL, HTML5, PHP, CSS, Node.JS, React, React Native (Android \& IOS), ASP.NET, Web services (WSDL, SVC, ASMX), Shell Script, PowerShell, LaTeX}

%---------------------------------------------------------
  \cvskill
    {Platforms}
    {Linux Server (Admin linux, PostgreSQL, MySQL, ...), Windows Server (DC, DHCP, Active Directory, IIS, Services Custom, ...), Cloud Computing (Google Cloud Platform, Microsoft Azure, AWS, OpenStack), Container (Docker, Kubernetes, OpenShift), Jenkins, Wireshark, Hortonworks \& Cloudera (Hadoop, Hive, Spark, Nifi, Cloudbreak, Ambari, Solr, Knox), Elastic Search, Power BI, Tableau, CMS, JDBC}

%---------------------------------------------------------
  \cvskill
    {Other} % Category
    {Git, Mercurial, CI/CD, JIRA, AutoCAD, 3DS Max, MAYA, Adobe Creative Cloud, Networks, Rapid7 (Metasploit, InsightVM)}
	
%---------------------------------------------------------
  \cvskill
    {Languages}
    {English, French}

%---------------------------------------------------------
\end{cvskills}

%-------------------------------------------------------------------------------
%	SECTION TITLE
%-------------------------------------------------------------------------------
\cvsection{Experiences}


%-------------------------------------------------------------------------------
%	CONTENT
%-------------------------------------------------------------------------------
\begin{cventries}

%---------------------------------------------------------
  \cventry
    {Consultant - Big Data Engineer, Data Scientist and Cloud Engineer}
    {Desjardins}
    {Montreal, QC \& Quebec, QC}
    {August 2018 - Present}
    {
      \begin{cvitems}
        \item {Define the solution to use to create a data lake for Desjardins.}
        \item {Development of the cloud environment to host the secured data lake solution.}
        \item {Cloud replication of an active directory and domain service (ldaps, kerberos).}
        \item {Install and configure a Hortonworks HDP and HDF solution now Cloudera CDP adapted to Desjardins.}
        \item {Implement ten pillars of PCI \& FIPS security standard to enable an end-to-end secure pipeline.}
        \item {Develop scripts to modify the product and allow it to work in the cloud while using internal infrastructures at the same time.}
        \item {Creating models and data flows.}
        \vspace{2mm}
        \begin{cvskills}
          \cvskill
          {Technologies}
    	  {Hadoop, Hortonworks, Azure, AWS, GCP, PostgreSQL, Rapid7, CI/CD (GitLab, Jenkins), Docker (OpenShift, Kubernetes)}
    	  \cvskill
          {Positions}
    	  {Big Data Engineer, Solution Architect, Cloud Engineer, Data Scientist}
        \end{cvskills}
        \begin{cvsubentries}
        \end{cvsubentries}
      \end{cvitems}
    }

%---------------------------------------------------------
  \cventry
    {Director - Artificial Intelligence, Big Data and Cloud Computing}
    {Progranova}
    {Laval, QC}
    {August 2017 - Present}
    {
      \begin{cvitems}
        \item {Manage and assist a team of several developers for the creation of Artificial Intelligence, Big Data and Cloud Computing projects. Agile project management and support.}
        \item {Show how to create value with often unused data.}
        \item {Design big data solution architectures for several types of use.}
		\item {Attend Big Data and Artificial Intelligence conferences and training in North America (Google, AWS, Azure, Hortonworks, Cloudera, CES)}
      \end{cvitems}
    }
	
%---------------------------------------------------------
  \cventry
    {Consultant - Artificial Intelligence and Advanced Analytics}
    {Raymond Chabot Grant Thornton}
    {Montreal, QC}
    {January 2018 - August 2018}
    {
      \begin{cvitems}
        \item {Acted as a consultant for the development of artificial intelligence and advanced analytics solutions for RCGT clients.}
        \item {Develop better data structures to help software perform better.}
		\item {Create development architectures to enable optimization of applications and infrastructure.}
		\item {Participate actively in artificial intelligence activities in Quebec.}
		\vspace{2mm}
        \begin{cvskills}
          \cvskill
          {Technologies}
    	  {Python, R, Keras, Tensorflow, Tableau, SQL, CI/CD, Docker}
    	  \cvskill
          {Positions}
    	  {AI Developer, Data Scientist, BI Developer, Technology Advisor}
        \end{cvskills}
		\begin{cvsubentries}
        \cvsubentry{CN Investment Division - Private Contract}{}
			{
				- Develop a system for classifying traded securities with data from a Bloomberg terminal. This system is able to classify thousands of securities based on analyzes made previously by fund managers. Investments of tens of millions have been made on emerging markets using this technology.
			}
		\cvsubentry{ARTM – Government Contract}{}
			{
				- Develop jobs to allow real time and archived view of the displacements of customers in the OPUS network. These big data jobs are used to answer several questions as KPIs in Tableau.
			}
		\cvsubentry{Chatbot — Internal Contract}{}
			{
				- Develop a tool able to automatically answer questions asked by internal accountant based on a database of Word and PowerPoint. This tool is able to search on the internet if it does not find the answer suggested by the model, by transferring the query to DialogFlow.
			}
		\cvsubentry{Montreal Taxi Office VS Uber — Government Contract}{}
			{
				- Use geospatial computations on data from Uber to determine the taxis pick-up and calculate approximate losses for Montreal taxis.
			}
		\cvsubentry{GPEC – Government Contract}{}
			{
				- Develop a platform to manage the talent of a municipality in the suburbs of Quebec City. This platform allows managers to have the right information on the skills required and acquired from their employees to allow better management. An artificial intelligence has been developed to read the information contained on different physical mediums.
			}
		\end{cvsubentries}
      \end{cvitems}
    }

%---------------------------------------------------------
  \cventry
    {Consultant - Full-Stack Cloud Developer}
    {CAE}
    {Montreal, QC}
    {September 2017 - December 2017}
    {
      \begin{cvitems}
        \item {Create whole new software to improve business productivity and business intelligence.}
        \item {Improve a PHP MediaWiki web application in terms of GUI and user experience by programming new CMS modules for the community and using Big Data platforms.}
		\item {Add multiple modules to existing programs to enable better integration with big data and lifecycle management.}
		\item {Support development teams in the design and estimation of added value.}
		\vspace{2mm}
        \begin{cvskills}
          \cvskill
          {Technologies}
    	  {C\#, HTML, CSS, JavaScript, SQL, Power BI, Web services, Node.JS, Azure, Elastic Search}
    	  \cvskill
          {Positions}
    	  {Full-Stack Developer, Cloud Engineer, BI Developer, Linux Admin, Windows Admin}
        \end{cvskills}
		\begin{cvsubentries}
                \cvsubentry{CAE Integrator}{}
		{
		    - Create a platform from scratch to read large sets of Business Intelligence data and generate Power Dashboard BI with Azure. The platform can also display the generated content in all the meetings or television rooms in the organization, even if the data is stored in another country. With this platform, someone can create a BI Dashboard in Montreal and send the content on a screen in the Tango Room located in Australia in less than 8 seconds. At all times the data is secure and controlled with the azure active directory.
		}
		\cvsubentry{PLM Toolbox (BOM Obsolete and Dynamic Server Functionality)}{}
		{
		    - Develop new features for older software developed by CAE in 2009 to help engineers in their life cycle management task. The BOM Obsolete module can scan an Oracle PLM database to extract all the parts that will become obsolete in the next 6 months. The engineer will also be informed of any necessary modifications to the related parts.\newline
            - Conversion of the PLM Toolbox solution to make the solution agnostic to its environment and thus be able to make the CI / CD.
		}
		\cvsubentry{WikiPEAK Extensions for CMS}{}
		{
		    - Create extension for a MediaWiki CMS. Implement a PHP search engine that uses supervising learning to display relevant results based on previous research. The search engine can also index all documents, images and videos (ElasticSearch).
		}
		\end{cvsubentries}
      \end{cvitems}
    }
	
%---------------------------------------------------------
  \cventry
    {IT Technician}
    {Geek Squad (Future Shop/Best Buy)}
    {St Jerome, QC}
    {June 2013 - March 2016}
    {
      \begin{cvitems}
        \item {Perform maintenance, virus removal and adware and computer repair.}
        \item {Manage difficult customers through exceptional customer service.}
        \item {Manage priorities to optimize delivery time.}
      \end{cvitems}
    }

%---------------------------------------------------------
\end{cventries}
%-------------------------------------------------------------------------------
%	SECTION TITLE
%-------------------------------------------------------------------------------
\cvsection{Professional Associations}

%-------------------------------------------------------------------------------
%	CONTENT
%-------------------------------------------------------------------------------
\begin{cventries}

%---------------------------------------------------------
  \cventry
    {VIP Member - Acquisition and Retention Board Committee}
    {The Quebec Association of IT Freelancers}
    {Montreal, QC}
    {December 2018 - Present}
    {
      \begin{cvitems}
        \item {Advise the board of directors in setting up services and training for young professionals who want to become an independent consultant.}
		\item {Writing and designing the landing page for the advertising campaign used to attract young professionals to become consultant.}
		\item {\textbf{NO : 23758}}
      \end{cvitems}
    }

%---------------------------------------------------------
  \cventry
    {Member}
    {The Quebec Association of IT Freelancers}
    {Montreal, QC}
    {June 2017 – December 2018}
    {
      \begin{cvitems}
		\item {\textbf{NO : 23758}}
      \end{cvitems}
    }

%---------------------------------------------------------
  \cventry
    {Member}
    {Genium360 (Québec engineering network)}
    {Montreal, QC}
    {September 2016 - Present}
    {
      \begin{cvitems}
		\item {\textbf{NO : 731117}}
      \end{cvitems}
    }

%---------------------------------------------------------
\end{cventries}
%-------------------------------------------------------------------------------
%	SECTION TITLE
%-------------------------------------------------------------------------------
\cvsection{Education}


%-------------------------------------------------------------------------------
%	CONTENT
%-------------------------------------------------------------------------------
\begin{cventries}

%---------------------------------------------------------
  \cventry
    {Bachelor of Software Engineering}
    {Concordia University}
    {Montreal, QC}
    {September 2016 - August 2018}
    {
      \begin{cvitems}
        \item {Member of the Major League of Hacking}
        \item {Mathematics and Method of Application}
        \item {Knowledge recognition for master's degrees received}
      \end{cvitems}
    }

%---------------------------------------------------------
\end{cventries}

\begin{cvhonors}

%---------------------------------------------------------
  \cvhonor
    {Machine Learning A-Z: Hands-On Python \& R In Data Science}
    {Udemy}
    {Online}
    {2017}
    
%---------------------------------------------------------
  \cvhonor
    {Google Cloud Platform Architect}
    {Google Canada}
    {Montreal, QC}
    {2017}

%---------------------------------------------------------
  \cvhonor
    {CS229 : Machine Learning}
    {Stanford University}
    {Online}
    {2017}

%---------------------------------------------------------
  \cvhonor
    {6.858 : Computer Systems Security}
    {MIT}
    {Online}
    {2017}

%---------------------------------------------------------
  \cvhonor
    {Introduction to Hortonworks}
    {Société Général}
    {Montreal, QC}
    {2017}
	
%---------------------------------------------------------
  \cvhonor
    {A+ Certification}
    {CompTIA}
    {Montreal, QC}
    {2014}

%---------------------------------------------------------
\end{cvhonors}
%-------------------------------------------------------------------------------
%	SECTION TITLE
%-------------------------------------------------------------------------------
\cvsection{Competitions}


%-------------------------------------------------------------------------------
%	CONTENT
%-------------------------------------------------------------------------------
\begin{cventries}

%---------------------------------------------------------
  \cventry
    {Team lead}
    {ConUHacks III}
    {Concordia University}
    {January 2018}
    {
      \begin{cvitems}
        \item {Develop a Time Attack on the Montreal metro ticket. Decryption of an Opus card to find the time 0 (January 1, 1990 00:00:00). Creation of an Android application to read and write an NFC card and inject our Time Attack.}
      \end{cvitems}
    }

%---------------------------------------------------------
  \cventry
    {Team lead}
    {YHack 2017}
    {Yale University}
    {November 2017}
    {
      \begin{cvitems}
        \item {Use Google's data to apply Machine Learning in the education sector. Our team decided to use Tensoflow to detect written or printed text to automatically create Google Forms' forms in Google Classroom. This technology helps teachers move their paper-based exercises to forms in Google Cloud.}
      \end{cvitems}
    }

%---------------------------------------------------------
\end{cventries}
%-------------------------------------------------------------------------------
%	SECTION TITLE
%-------------------------------------------------------------------------------
\cvsection{Interests \& other knowledge}


%-------------------------------------------------------------------------------
%	CONTENT
%-------------------------------------------------------------------------------
\begin{cventries}

%---------------------------------------------------------
  \cventry
    {}
    {Corporate Administration \& Project Management}
    {}
    {}
    {
      \begin{cvitems} 
        \item {I know how to manage a business, from the process of incorporation to the trademark process, from accounting to tax reporting, and from drafting contracts to selecting employees.}
		\item {I also had the chance to manage projects during some of my contract. So I had the chance to test several agile methodologies and develop my skills toward this direction.}
      \end{cvitems}
    }

%---------------------------------------------------------
  \cventry
    {}
    {Experience abroad}
    {}
    {}
    {
      \begin{cvitems}
        \item {Electronic Entertainment Expo (E3) in Los Angeles (2016) as a media member for my own company. In addition to a visit to Google headquarters for the presentation of a new product line and development training (Google IO 2016).}
      \end{cvitems}
    }

%---------------------------------------------------------
  \cventry
    {}
    {Investment Portfolio Management - Theroux Investissement}
    {}
    {}
    {
      \begin{cvitems}
        \item {I have a very good knowledge of the financial markets and how to apply artificial intelligence techniques to increase the yield with fast trading algorithms.}
		\item {I had the chance to put my financial expertise into practice during the CN Investment Division project by working on the implementation of a neural network on data from a Bloomberg terminal.}
      \end{cvitems}
    }

%---------------------------------------------------------
\end{cventries}

%-------------------------------------------------------------------------------
\end{document}