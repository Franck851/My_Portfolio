%!TEX TS-program = xelatex
%!TEX encoding = UTF-8 Unicode

%-------------------------------------------------------------------------------
% CONFIGURATIONS
%-------------------------------------------------------------------------------
% A4 paper size by default, use 'letterpaper' for US letter
\documentclass[11pt, a4paper]{awesome-cv}

% Configure page margins with geometry
\geometry{left=1.4cm, top=.8cm, right=1.4cm, bottom=1.8cm, footskip=.5cm}

% Specify the location of the included fonts
\fontdir[fonts/]

% Color for highlights
\colorlet{awesome}{awesome-blue}

% Set false if you don't want to highlight section with awesome color
\setbool{acvSectionColorHighlight}{true}

% If you would like to change the social information separator from a pipe (|) to something else
\renewcommand{\acvHeaderSocialSep}{\quad\textbar\quad}

%-------------------------------------------------------------------------------
%	PERSONAL INFORMATION
%-------------------------------------------------------------------------------
% Available options: circle|rectangle,edge/noedge,left/right
\photo[rectangle,noedge,right]{profile}
\name{François}{Théroux}
\position{Consultant - Intelligence Artificielle, Big Data et Cloud Computing}

\mobile{+1 (450) 400-1040}
\email{ftheroux@progranova.com}
\homepage{www.progranova.com}
\linkedin{ftheroux}

\quote{``Ma passion: Stocker, visualiser et utiliser la donnée pour créer de la valeur dans votre entreprise"}

%-------------------------------------------------------------------------------
\begin{document}

% Give optional argument to change alignment(C: center, L: left, R: right)
\makecvheader[L]

% Print the footer with 3 arguments(<left>, <center>, <right>)
\makecvfooter
  {Progranova - Copyright \textcopyright 2019}
  {François Théroux~~~·~~~Curriculum Vitae}
  {\thepage}

%-------------------------------------------------------------------------------
%	CV/RESUME CONTENT
%-------------------------------------------------------------------------------
%-------------------------------------------------------------------------------
%	SECTION TITLE
%-------------------------------------------------------------------------------
\cvsection{Compétences}


%-------------------------------------------------------------------------------
%	CONTENT
%-------------------------------------------------------------------------------
\begin{cvskills}

%---------------------------------------------------------
  \cvskill
    {Programmation}
	{Python, R, Java, Scala, C\#, C/C++, JavaScript, SQL, NoSQL, HTML5, PHP, CSS, Node.JS, React, React Native (Android \& IOS), ASP.NET, Web services (WSDL, SVC, ASMX), Shell Script, PowerShell, LaTeX}

%---------------------------------------------------------
  \cvskill
    {Plateformes}
    {Linux Server (Admin linux, PostgreSQL, MySQL, ...), Windows Server (DC, DHCP, Active Directory, IIS, Services Custom, ...), Cloud Computing (Google Cloud Platform, Microsoft Azure, AWS, OpenStack), Container (Docker, Kubernetes, OpenShift), Jenkins, Wireshark, Hortonworks \& Cloudera (Hadoop, Hive, Spark, Nifi, Cloudbreak, Ambari, Solr, Knox), Elastic Search, Power BI, Tableau, CMS, JDBC}

%---------------------------------------------------------
  \cvskill
    {Autre} % Category
    {Git, Mercurial, CI/CD, JIRA, AutoCAD, 3DS Max, MAYA, Adobe Creative Cloud, Réseau, Rapid7 (Metasploit, InsightVM)}
	
%---------------------------------------------------------
  \cvskill
    {Langues}
    {Français, Anglais}

%---------------------------------------------------------
\end{cvskills}

%-------------------------------------------------------------------------------
%	SECTION TITLE
%-------------------------------------------------------------------------------
\cvsection{Expériences}


%-------------------------------------------------------------------------------
%	CONTENT
%-------------------------------------------------------------------------------
\begin{cventries}

%---------------------------------------------------------
  \cventry
    {Consultant - Big Data Engineer, Data Scientist et Cloud Engineer}
    {Desjardins}
    {Montréal, QC \& Québec, QC}
    {Août 2018 - Présent}
    {
      \begin{cvitems}
        \item {Définir la solution à utiliser pour créer un lac de données chez Desjardins.}
        \item {Développement de l'environnement infonuagique pour accueillir une solution sécurisée.}
        \item {Replication cloud d'un active directory, domain service (ldaps, kerberos).}
        \item {Installer et configurer une solution Hortonworks HDP et HDF maintenant devenue Cloudera adapter a Desjardins.}
        \item {Implémenter dix piliers de sécurité PCI \& FIPS pour permettre un pipeline sécurise de bout en bout.}
        \item {Développer des scripts pour modifier le produit et lui permettre de fonctionner dans le cloud tout en utilisant des infrastructures internes.}
        \item {Création de modèles et de flux de données.}
        \vspace{2mm}
        \begin{cvskills}
          \cvskill
          {Technologies}
    	  {Hadoop, Hortonworks, Azure, AWS, GCP, PostgreSQL, Rapid7, CI/CD (GitLab, Jenkins), Docker (OpenShift, Kubernetes)}
    	  \cvskill
          {Postes}
    	  {Big Data Engineer, Architect de solutions, Cloud Engineer, Data Scientist}
        \end{cvskills}
        \begin{cvsubentries}
        \end{cvsubentries}
      \end{cvitems}
    }

%---------------------------------------------------------
  \cventry
    {Directeur - Intelligence Artificielle, Big Data et Cloud Computing}
    {Progranova}
    {Laval, QC}
    {Août 2017 - Présent}
    {
      \begin{cvitems}
        \item {Gérer et assister une équipe de plusieurs développeurs pour la création de projets d'intelligence artificielle, Big Data et Cloud Computing. Gestion de projet Agile et accompagnement.}
        \item {Montrer comment créer de la valeur avec des données bien souvent inutilisées.}
        \item {Concevoir des architectures de solution big data pour plusieurs types d'utilisation.}
		\item {Assister à des conférences et formations Big Data et l'intelligence artificielle en Amérique du Nord (Google, AWS, Azure, Hortonworks, Cloudera, CES)}
      \end{cvitems}
    }
	
%---------------------------------------------------------
  \cventry
    {Consultant - Intelligence Artificielle et Analytique Avancé}
    {Raymond Chabot Grant Thornton}
    {Montréal, QC}
    {Janvier 2018 - Août 2018}
    {
      \begin{cvitems}
        \item {Agir à titre de consultant pour le développement de solutions d’intelligence artificielle et d’analytique avancé auprès des clients de RCGT.}
        \item {Développer de meilleures structures de données pour aider les logiciels à mieux fonctionner.}
		\item {Créer des architectures de développement pour permettre une optimisation des applications et des infrastructures.}
		\item {Participer activement aux activités sur l’intelligence artificielle présente sur le territoire québécois.}
		\vspace{2mm}
        \begin{cvskills}
          \cvskill
          {Technologies}
    	  {Python, R, Keras, Tensorflow, Tableau, SQL, CI/CD, Docker}
    	  \cvskill
          {Postes}
    	  {AI Developer, Data Scientist, Bi Developper, Conseiller en virage technologique}
        \end{cvskills}
		\begin{cvsubentries}
        \cvsubentry{CN Investment Division – Mandat Privé}{}
			{
				- Développer un système permettant de classifier des titres cotés en bourse avec les données provenant d'un terminal Bloomberg. Ce système permet de faire l'analyse de milliers de titres basée sur des analyses faites précédemment par des gestionnaires de fonds. Des investissements de plusieurs dizaines de millions ont été réalisés sur les marchés émergents en utilisant cette technologie.
			}
		\cvsubentry{ARTM – Mandat Gouvernemental}{}
			{
				- Développer des jobs pour permettre une vue en temps réel et archivé des déplacements des usagers sur le réseau OPUS. Ces jobs big data permettent de répondre à plusieurs questions représentées sous forme de KPI dans Tableau.
			}
		\cvsubentry{Chatbot — Mandat Interne}{}
			{
				- Développer un outil permettant la réponse automatique aux questions posées par les utilisateurs internes en certification basés sur une base de données énorme. L’outil est en mesure de rechercher sur internet s’il ne trouve pas la réponse suggérée par le modèle, en transférant la requête sur DialogFlow.
			}
		\cvsubentry{Bureau du taxi VS Uber — Mandat Gouvernemental}{}
			{
				- Utilisez les calculs géospatiaux sur des données provenant d’Uber pour déterminer la zone de prise en charge des taxis et ainsi calculer les pertes approximatives pour les taxis de Montréal.
			}
		\cvsubentry{GPEC – Mandat Gouvernemental}{}
			{
				- Développer une plateforme permettant la gestion des talents d'une municipalité en banlieue de Québec. Cette plateforme permet aux gestionnaires d’avoir l’heure juste sur les compétences requises et acquises de leurs employés pour permettre une meilleure gestion de ceux-ci. Une intelligence artificielle a dû être développée pour permettre la lecture complète des informations contenues sur différents médiums physiques.
			}
		\end{cvsubentries}
      \end{cvitems}
    }

%---------------------------------------------------------
  \cventry
    {Consultant - Développeur Full-Stack Cloud}
    {CAE}
    {Montréal, QC}
    {Septembre 2017 - Décembre 2017}
    {
      \begin{cvitems}
        \item {Créer de nouveaux logiciels en entier pour améliorer la productivité de l’entreprise et l’intelligence d’affaires.}
        \item {Améliorer une application web PHP MediaWiki sur le plan de l’interface graphique et l’expérience utilisateur en programmant de nouveaux modules CMS pour la communauté et en utilisant des plateformes Big Data.}
		\item {Ajouter plusieurs modules à des programmes existants pour permettre une meilleure intégration avec le big data et la gestion du cycle de vie.}
		\item {Soutenir des équipes développement dans la conception et l'estimation de la valeur ajoutée.}
		\vspace{2mm}
        \begin{cvskills}
          \cvskill
          {Technologies}
    	  {C\#, HTML, CSS, JavaScript, SQL, Power BI, Web services, Node.JS, Azure, Elastic Search}
    	  \cvskill
          {Postes}
    	  {Full-Stack Developer, Cloud Engineer, BI Developer, Linux Admin, Windows Admin}
        \end{cvskills}
		\begin{cvsubentries}
        \cvsubentry{CAE Integrator}{}
		{
		    - Créé de toute pièce d'une plateforme pour lire de grands ensembles de données de Business Intelligence et générer des Dashboard Power BI avec Azure. La plateforme peut également afficher le contenu généré dans toutes les salles de réunions ou télévision de l’organisation, même si les données sont stockées dans un autre pays. Avec cette plateforme, quelqu’un peut créer un Dashboard BI à Montréal et envoyer le contenu sur un écran dans la salle Tango situé en Australie en moins de 8 secondes. En tout temps la donnée est sécurisé et contrôlé avec l’azure active directory.
		}
		\cvsubentry{PLM Toolbox (BOM Obsolete et Fonctionnalité de serveur dynamique)}{}
		{
		    - Développer de nouvelles fonctionnalités pour un ancien logiciel développé par CAE en 2009 pour aider les ingénieurs dans leur tâche de gestion du cycle de vie des produits. Le module BOM Obsolete peut scanner une base de données PLM pour extraire toutes les pièces qui deviendront obsolètes dans les 6 prochains mois. L'ingénieur sera aussi informé de toutes modifications nécessaires des pièces connexes.\newline
            - Conversion de la solution PLM Toolbox pour rendre la solution agnostique à son environnement et ainsi pouvoir faire du CI/CD.
		}
		\cvsubentry{Extensions WikiPEAK pour CMS}{}
		{
		    - Créations d'extensions pour un CMS MediaWiki. Implémenter un moteur de recherche PHP qui utilise du supervise learning pour afficher des résultats pertinents basés sur des recherches précédentes. Le moteur de recherche peut également indexer tous les documents, images et vidéos (ElasticSearch).
		}
		\end{cvsubentries}
      \end{cvitems}
    }
	
%---------------------------------------------------------
  \cventry
    {Technicien informatique}
    {Geek Squad (Future Shop/Best Buy)}
    {St Jérôme, QC}
    {Juin 2013 - Mars 2016}
    {
      \begin{cvitems}
        \item {Effectuer des maintenances, suppression de virus et d’adware et réparation d’ordinateurs.}
        \item {Gérer de la clientèle difficile par l'entremise d'un service à la clientèle exceptionnel.}
        \item {Gérer les priorités pour permettre une optimisation du temps de livraison.}
      \end{cvitems}
    }

%---------------------------------------------------------
\end{cventries}
%-------------------------------------------------------------------------------
%	SECTION TITLE
%-------------------------------------------------------------------------------
\cvsection{Associations professionnelles}

%-------------------------------------------------------------------------------
%	CONTENT
%-------------------------------------------------------------------------------
\begin{cventries}

%---------------------------------------------------------
  \cventry
    {Membre VIP - Comité au conseil d'acquisition et rétention}
    {Association Québécoise des Informaticiennes et Informaticiens Indépendants}
    {Montréal, QC}
    {Décembre 2018 - Présent}
    {
      \begin{cvitems}
        \item {Conseiller le conseil d'administration dans la mise en place de services et formations pour les jeunes professionnels qui veulent faire le saut comme consultant indépendant.}
		\item {Rédaction et conception de la landing page pour la campagne publicitaire visant a attirer les jeunes professionnels vers la consultation.}
		\item {\textbf{NO : 23758}}
      \end{cvitems}
    }

%---------------------------------------------------------
  \cventry
    {Membre}
    {Association Québécoise des Informaticiennes et Informaticiens Indépendants}
    {Montréal, QC}
    {Juin 2017 – Décembre 2018}
    {
      \begin{cvitems}
		\item {\textbf{NO : 23758}}
      \end{cvitems}
    }

%---------------------------------------------------------
  \cventry
    {Membre}
    {Genium360 (Réseau des ingénieurs du Québec)}
    {Montréal, QC}
    {2016 - Présent}
    {
      \begin{cvitems}
		\item {\textbf{NO : 731117}}
      \end{cvitems}
    }

%---------------------------------------------------------
\end{cventries}
%-------------------------------------------------------------------------------
%	SECTION TITLE
%-------------------------------------------------------------------------------
\cvsection{Éducation}


%-------------------------------------------------------------------------------
%	CONTENT
%-------------------------------------------------------------------------------
\begin{cventries}

%---------------------------------------------------------
  \cventry
    {Baccalauréat en Génie logiciel}
    {Concordia University}
    {Montréal, QC}
    {Septembre 2016 - Août 2018}
    {
      \begin{cvitems}
        \item {Membre de la Major League of Hacking}
        \item {Mathématique et Méthode d’application}
        \item {Reconnaissance des acquis pour maitrise reçus}
      \end{cvitems}
    }

%---------------------------------------------------------
\end{cventries}

\begin{cvhonors}

%---------------------------------------------------------
  \cvhonor
    {Machine Learning A-Z: Hands-On Python \& R In Data Science}
    {Udemy}
    {En ligne}
    {2017}
    
%---------------------------------------------------------
  \cvhonor
    {Google Cloud Platform Architect}
    {Google Canada}
    {Montréal, QC}
    {2017}

%---------------------------------------------------------
  \cvhonor
    {CS229 : Machine Learning}
    {Stanford University}
    {En ligne}
    {2017}

%---------------------------------------------------------
  \cvhonor
    {6.858 : Computer Systems Security}
    {MIT}
    {En ligne}
    {2017}

%---------------------------------------------------------
  \cvhonor
    {Introduction to Hortonworks}
    {Société Général}
    {Montréal, QC}
    {2017}
	
%---------------------------------------------------------
  \cvhonor
    {A+ Certification}
    {CompTIA}
    {Montréal, QC}
    {2014}

%---------------------------------------------------------
\end{cvhonors}
%-------------------------------------------------------------------------------
%	SECTION TITLE
%-------------------------------------------------------------------------------
\cvsection{Compétitions}


%-------------------------------------------------------------------------------
%	CONTENT
%-------------------------------------------------------------------------------
\begin{cventries}

%---------------------------------------------------------
  \cventry
    {Chef d’équipe}
    {ConUHacks III}
    {Concordia University}
    {Janvier 2018}
    {
      \begin{cvitems}
        \item {Développer un Time attack sur les passes l'occasionnel de la STM. Décryptage des cartes de métro pour déterminer le temps zéro du sytème Opus (1er janvier 1990 00:00:00). Création d'une application Android pour lire et écrire une carte NFC et injecter notre Time attack.}
      \end{cvitems}
    }

%---------------------------------------------------------
  \cventry
    {Chef d’équipe}
    {YHack 2017}
    {Yale University}
    {Novembre 2017}
    {
      \begin{cvitems}
        \item {Utiliser les données de Google pour appliquer le Machine Learning dans le milieu de l’éducation. Notre équipe a décidé d’utiliser Tensoflow pour détecter du texte écrit ou imprimé afin de créer automatiquement des formulaires Google Forms dans Google Classroom. Cette technologie aide les enseignants à déplacer leurs exercices papier vers des formulaires dans le nuage de Google.}
      \end{cvitems}
    }

%---------------------------------------------------------
\end{cventries}
\newpage
%-------------------------------------------------------------------------------
%	SECTION TITLE
%-------------------------------------------------------------------------------
\cvsection{Intérêts \& autres connaissances}


%-------------------------------------------------------------------------------
%	CONTENT
%-------------------------------------------------------------------------------
\begin{cventries}

%---------------------------------------------------------
  \cventry
    {}
    {Administration d’entreprise \& Gestion de projet}
    {}
    {}
    {
      \begin{cvitems} 
        \item {Je sais gérer une entreprise, du processus d’incorporation au processus relatif aux marques de commerces, de la comptabilité jusqu’aux rapports fiscaux et de la rédaction des contrats jusqu’à la sélection d’employés.}
		\item {J'ai également eu la chance de gérer des projets lors de mes mandats. J'ai donc eu la chance de tester plusieurs méthodologies agiles et développer mes compétences en ce sens.}
      \end{cvitems}
    }

%---------------------------------------------------------
  \cventry
    {}
    {Expérience à l’étranger}
    {}
    {}
    {
      \begin{cvitems}
        \item {Electronic Entertainment Expo (E3) à Los Angeles (2016) en tant que membre média pour ma propre entreprise. En plus d’une visite du siège de Google pour la présentation d’une nouvelle gamme de produits et formation de développement (Google IO 2016).}
      \end{cvitems}
    }

%---------------------------------------------------------
  \cventry
    {}
    {Gestion de portefeuille d’investissement - Theroux Investissement}
    {}
    {}
    {
      \begin{cvitems}
        \item {J’ai une très bonne connaissance des marchés financiers et de la façon d’appliquer des techniques d’intelligence artificielle pour permettre d'augmenter le rendement avec des algorithmes de fast trading.}
		\item {J’ai eu la chance de mettre en pratique cette expertise en travaillant sur la mise en place d’un neural network sur les données d'un terminal Bloomberg dans le cadre d’un projet pour le CN Investment Division.}
      \end{cvitems}
    }

%---------------------------------------------------------
\end{cventries}

%-------------------------------------------------------------------------------
\end{document}