%-------------------------------------------------------------------------------
%	SECTION TITLE
%-------------------------------------------------------------------------------
\cvsection{Compétitions}


%-------------------------------------------------------------------------------
%	CONTENT
%-------------------------------------------------------------------------------
\begin{cventries}

%---------------------------------------------------------
  \cventry
    {Chef d’équipe}
    {ConUHacks III}
    {Concordia University}
    {Janvier 2018}
    {
      \begin{cvitems}
        \item {Développer un Time attack sur les passes l'occasionnel de la STM. Décryptage des cartes de métro pour déterminer le temps zéro du sytème Opus (1er janvier 1990 00:00:00). Création d'une application Android pour lire et écrire une carte NFC et injecter notre Time attack.}
      \end{cvitems}
    }

%---------------------------------------------------------
  \cventry
    {Chef d’équipe}
    {YHack 2017}
    {Yale University}
    {Novembre 2017}
    {
      \begin{cvitems}
        \item {Utiliser les données de Google pour appliquer le Machine Learning dans le milieu de l’éducation. Notre équipe a décidé d’utiliser Tensoflow pour détecter du texte écrit ou imprimé afin de créer automatiquement des formulaires Google Forms dans Google Classroom. Cette technologie aide les enseignants à déplacer leurs exercices papier vers des formulaires dans le nuage de Google.}
      \end{cvitems}
    }

%---------------------------------------------------------
\end{cventries}